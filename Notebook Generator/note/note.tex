\begin{itemize}
	\item  Hall’s Marriage Theorem In a bipartite
graph G = A $\cup$ B, a matching saturating A exists iff $|N(S)| >= |S|$ for all S $\subset$ A
	\item - $c'((s', v)) = \sum_{u \in V} d((u, v))$ for each edge $(s', v)$.\\
- $c'((v, t')) = \sum_{w \in V} d((v, w))$ for each edge $(v, t')$.\\
- $c'((u, v)) = c((u, v)) - d((u, v))$ for each edge $(u, v)$ in the old network.\\
- $c'((t, s)) = \infty$
\item Every positive integer has a unique representation as a sum of Fibonacci numbers such that no two numbers are equal or
consecutive Fibonacci numbers. Pythagorean triples $(n^2-m^2 , 2nm , n^2 + m^2) $ where $ 0<m<n$ , $n$ and $m$ are  coprime and at least one of n and m is even.  Each box may contain at most one ball, and in addition, no two
adjacent boxes may both contain a ball.${n-k+1}\choose{n-2k+1}$
\item The Catalan number $C_n$ equals the number of valid parenthesis expressions
that consist of $n$ left parentheses and $n$ right parentheses.
$1, 2, 5, 14, 42, 132, 429, 1430, 4862, 16796$\\
$C_n = \frac{1}{n+1} {{2n}\choose{n}}$
there are $C_n$ binary trees of n nodes,
there are $C_{n-1}$ rooted trees of n nodes
\item  The number of derangements of elements $\left\{1,2,...,n\right\}$,
i.e., permutations where no element remains in its original place.\\
$f (n) = (n-1)(f(n-2)+ f (n-1)) , n > 2$
There are $n-1$ ways to choose an element x that
replaces the element 1.
Option 1: We also replace the element x with the element 1. After this, the
remaining task is to construct a derangement of $n-2$ elements.
Option 2: We replace the element x with some other element than 1. Now we
have to construct a derangement of $n-1$ element, because we cannot replace the
element x with the element 1, and all other elements must be changed.\\
Another formula
$F(n) = n!-\binom{n}{1} \times (n-1)!+ \binom{n}{2} \times (n-2)!-..…+ (-1)^k \times \binom{n}{k} \times (n-k)!+….+(-1)^n$
\item $\sum_{m = 0}^n \binom{m}{k} = \binom{n + 1} {k + 1}$~~~~~~~~~~~~~~~ $\sum_{k = 0}^m  \binom {n + k} k = \binom {n + m + 1} m$ ~~~~~~~~~~~~~~~~~~~ ${\binom n 0}^2 + {\binom n 1}^2 + \cdots + {\binom n n}^2 = \binom {2n} n$
\item $1 \binom n 1 + 2 \binom n 2 + \cdots + n \binom n n = n 2^{n-1}$ ~~~~~~~~~~~~~~~~~~~~~~~~~~ $\binom n 0 + \binom {n-1} 1 + \cdots + \binom {n-k} k + \cdots + \binom 0 n = F_{n+1}$
\item The number of necklaces of n pearls, where
each pearl has m possible colors $\sum_{i=0}^{n-1} \frac{m^{gcd(i,n)}}{n}$
\item Number of bracket sequence with n open and m close brackets starts with value k
C(m+n,m)- C(m+n,m-k-1)
\item The area of the polygon is
$a+ b/2 - 1 $
where a is the number of integer points inside the polygon and b is the number
of integer points on the boundary of the polygon.
\item Number of lattice point between 2 points 
$gcd( abs(p1.x- p2.x) ,abs(p1.y- p2.y) )-1$
\item A useful technique related to Manhattan distances is to rotate all coordinates
$45$ degrees so that a point $(x, y)$ becomes $(x+ y, y- x)$.
$|x1-x2| + |y1-y2| = $ max$(|x1' - x2'| , |y1' - y2'|)$
\item Size of maximum matching in a bipartite graph is equal to the size of its minimum vertex cover, 
and the minimum vertex cover can be reconstructed after finding the maximum matching. 
If we remove a vertex from the minimum vertex cover, the size of the minimum vertex cover 
of the remaining graph is reduced by 1, so the size of the maximum matching is reduced by 1 as well.
It means that we can always choose to remove a vertex from the minimum vertex cover we found. By the way,
it also proves that it's always possible to remove a vertex from a bipartite graph so the size of the
maximum matching decreases by 1 (obviously, if it's not 0 already).
\item Zigzag numbers are as follows $1, 1, 1, 2, 5, 16, 61, 272, 1385, 7936, 50521 …… $ \\
$a(n+1) = (\sum_{k=0}^{n} {n\choose k} * a(k) * a(n-k))/2$\\
$E(n,k) = E(n,k-1) + E(n-1,n-k)$ if $k\geq n$ or $k<1$ $E(n,k) = 0$
\item The first few values of the partition function, starting with $p(0) = 1$, are:\\
$1, 1, 2, 3, 5, 7, 11, 15, 22, 30, 42, 56, 77, 101, 135, 176, 231, 297, 385, 490, 627, 792, 1002, 1255, 1575, 1958,... $
\item St kind 2 :  $ A(n,k) = A(n-1,k-1) + k*A(n-1,k)$ ~~~~~~~~~~ St kind 1 :  $ A(n,k) = A(n-1,k-1) + (n-1)*A(n-1,k)$
\item Eulerian number with k ascents : $A(n,k) = (k+1)*A(n-1,k) + (n-k)*A(n-1,k-1)$
\item Mo on tree : 1: P = u then [ST(u),ST(v)] , 2:[EN(u),ST(v)] + [ST(P),ST(P)] 
\end{itemize}
\begin{multicols}{3}
St kind 2: 1\\
0,1\\
0,1,1\\
0,1,3,1\\
0,1,7,6,1\\
0,1,15,25,10,1\\
0,1,31,90,65,15,1\\
0,1,63,301,350,140,21,1\\
St kind 1: 1\\
0,1\\
0,1,1\\
0,2,3,1\\
0,6,11,6,1\\
0,24.50,35,10,1\\
0,120,274,225,85,15,1\\
0,720,1764,1624,735,175,21,1\\
Euler number: 1 \\
1,0\\
1,1,0\\
1,4,1,0\\
1,11,11,1,0\\
1,26,66,26,1,0\\
1,57,302,302,57,1,0\\
\end{multicols}